\documentclass{article}
\usepackage[margin=1in]{geometry}


\title{Euler’s Identity}
\author{Amar Nath Singh NA21B005}
\date{June 2022}

\begin{document}

\maketitle

\section{Euler’s Identity}
\begin{equation}
    e^{i\pi} + 1 = 0
\end{equation}
\\
A very famous equation, Euler’s identity relates the seemingly random values of pi, e, and the square root of -1. It is considered by many to be the most beautiful equation in mathematics.

A more general formula is

\begin{equation}
    e^{i x} = \cos x + i \sin x
\end{equation}
\\
When x = $\pi$ , the value of $\cos$ x & is -1, while  i$\sin x$  is 0, resulting in Euler’s identity, as -1 + 1 = 0.\\ \\
\begin{center}
\begin{tabular}{ |l|l| } 
 \hline
 $\pi$ & The number $\pi$ is a mathematical constant that is approximately equal to 3.14159. \\  
 $e$ & It is the base of the natural logarithms, approximately equal to 2.71828.\\
 $i$ & The value of i is $\sqrt{-1}$.\\
 $\cos$ & cosine are trigonometric functions of an angle.\\
 $\sin$ & sine are trigonometric functions of an angle.\\
 \hline
\end{tabular}
\end{center}
\end{document}
